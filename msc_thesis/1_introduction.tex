\section{Introduction}
\subsection{Motivation}
The polyadenylation signal(PAS) is part of the 3' untranslated region(UTR) of almost every messenger RNA. It is the binding site for a protein complex that causes the formation of the polyadenylation tail(poly(A)-tail). This tail of adenosines prevents the degradation of the mRNA in the cytoplasm effectively affecting the half-life of a mRNA and thereby raising the chance to be translated into a protein by the ribosome. As such, variants that have an impact on the efficiency of the PAS are potentially pathogenic due to their interference with the gene expression. Chen et al. wrote a review about diseases caused by PAS-affecting mutations. [REFERENCE] 

Prediction of polyadenylation sites has been attempted several times with moderate success in the last twenty years. [REFERENCES/EXAMPLES] Seeing that we cannot predict PASs with high accuracy prompts the question if we have enough knowledge about the structure of polyadenylation sites. Analyzing the secondary structure of the 3' UTR might give us more insight and allows the use of more characteristics for the prediction.

Accurately identifying functional PASs allows for better detection of disease causing mutations.  
%Kann woanders hin:
%[The PAS is a well-conserved six nucleotide long motif located usually ten to thirty bases before the cleavage site. [REFERENCE] Knowing this, it is possible to detect a PAS by using a naive search in a mRNA database. However, there are several problems with this approach. Assuming all four bases are evenly distributed throughout the DNA the same hexamer appears about every 4000 bases making the motif sequence a common occurrence.] 
\subsection{Background}
\subsubsection{History}
Polyadenylation (poly(A)) is the process of adding adenosines to the 3' end of precursor mRNA (pre-mRNA), i.e. messenger RNA that has been recently transcribed and has yet to undergo any post-transcription steps like splicing or polyadenylation.

The enzyme polyadenylate polymerase which catalyzes the reaction was first discovered in 1960. \cite{pmid13819354} But the protein's function was revealed about a decade later. \cite{pmid5288383} At that time it was still a mystery why most mRNAs had polyadenylated tails. 

In the early 1980s the role of the poly(A)-tail was largely elucidated: nuclear export, translation and stability of the mature mRNA. (reviewed in \cite{pmid6111419}). 

It took another decade to discover the two primary proteins that mediate the 3' end cleavage and polyadenylation, namely cleavage and polyadenylation specificity factor (CPSF) and cleavage stimulation factor (CStF). The number of participating proteins has since increased and recent publications mention several dozen cis-acting and auxiliary proteins or polypeptides working together in a large complex. \cite{pmid23774734} 

2008 it became clear that most genes have more than one polyadenylation site, resulting in transcripts with different lengths but identical coding sequence. \cite{pmid18411206} The new finding was termed alternative polyadenylation (APA). Since the 3' untranslated Region (3' UTR) has many binding sites for mRNA-repressive micro RNAs (miRNA), APA is taking part in the gene expression. \cite{pmid18566288}
mRNA is a biomolecule generated by the enzyme RNA polymerase during the DNA transcription. The primary role of mRNA is to transport the genetic information stored by the DNA to the ribosome in the cytoplasm. The ribosome then translates the mRNA into a protein. 
\subsubsection{messenger ribonucleic acid (mRNA)}
\subsubsection{Polyadenylation}

