\section{Introduction}

\subsection{messenger ribonucleic acid (mRNA)}
mRNA is a biomolecule generated by the enzyme RNA polymerase during the DNA transcription. The primary role of mRNA is to transport the genetic information stored by the DNA to the ribosome in the cytoplasm. The ribosome then translates the mRNA into a protein. 

\subsection{History}
Polyadenylation (poly(A)) is the process of adding adenosines to the 3' end of precursor mRNA (pre-mRNA), i.e. messenger RNA that has been recently transcribed and has yet to undergo any post-transcription steps like splicing or polyadenylation.

The enzyme polyadenylate polymerase which catalyzes the reaction was first discovered in 1960. \cite{pmid13819354} But the protein's function was revealed about a decade later. \cite{pmid5288383} At that time it was still a mystery why most mRNAs had polyadenylated tails. 

In the early 1980s the role of the poly(A)-tail was largely elucidated: nuclear export, translation and stability of the mature mRNA. (reviewed in \cite{pmid6111419}). 

It took another decade to discover the two primary proteins that mediate the 3' end cleavage and polyadenylation, namely cleavage and polyadenylation specificity factor (CPSF) and cleavage stimulation factor (CStF). The number of participating proteins has since increased and recent publications mention several dozen cis-acting and auxiliary proteins or polypeptides working together in a large complex. \cite{pmid23774734} 

2008 it became clear that most genes have more than one polyadenylation site, resulting in transcripts with different lengths but identical coding sequence. \cite{pmid18411206} The new finding was termed alternative polyadenylation (APA). Since the 3' untranslated Region (3' UTR) has many binding sites for mRNA-repressive micro RNAs (miRNA), APA is taking part in the gene expression. \cite{pmid18566288}
